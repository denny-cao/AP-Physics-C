%++++++++++++++++++++++++++++++++++++++++
\documentclass[article, 11pt]{article}
\usepackage{float}
\usepackage{setspace}
\usepackage{tabu} % extra features for tabular environment
\usepackage{amsmath}  % improve math presentation
\usepackage{graphicx} % takes care of graphic including machinery
\usepackage[margin=1in]{geometry} % decreases margins
\usepackage{cite} % takes care of citations
\usepackage[final]{hyperref} % adds hyper links inside the generated pdf file
\usepackage{tikz}
\usepackage{caption} 
\usepackage{fancyhdr}
\usepackage{amssymb} % symbols like /therefore
\usepackage{amsthm} % proofs
\usepackage{enumerate} % lettered lists
\usepackage{mathtools} % macros
\usepackage{hyperref} % hyperlinks
\usetikzlibrary{scopes}
% \usepackage{xcolor} \pagecolor[rgb]{0.12549019607,0.1294117647,0.13725490196} \color[rgb]{0.82352941176,0.76862745098,0.62745098039} % dark theme
\theoremstyle{definition}
\newtheorem{example}{Example}[section]
\newtheorem*{remark}{Remark}
\newtheorem{theorem}{Theorem}[section]
\newtheorem{definition}{Definition}[section]
\newtheorem{corollary}{Corollary}[section]
\hypersetup{
	colorlinks=false,      % false: boxed links; true: colored links
	linkcolor=blue,        % color of internal links
	citecolor=blue,        % color of links to bibliography
	filecolor=magenta,     % color of file links
	urlcolor=blue         
}
\usepackage{titling}
\renewcommand\maketitlehooka{\null\mbox{}\vfill}
\renewcommand\maketitlehookd{\vfill\null}
\usepackage{siunitx} % units
\usepackage{verbatim} 
\newcommand{\studyTitle}{Study Guide} 
\newcommand{\class}{AP Physics C: Mechanics}
\newcommand{\professor}{Mr. Perkins}
\newcommand{\name}{Denny Cao}
\newcommand{\examDate}{January 17, 2023}
\pagestyle{fancy}
\fancyhf{}% clears all header and footer fields
\fancyfoot[C]{--~\thepage~--}
\renewcommand*{\headrulewidth}{0.4pt}
\renewcommand*{\footrulewidth}{0pt}
\lhead{\name}
\chead{\leftmark}
\rhead{\professor}


\fancypagestyle{plain}{%
  \fancyhf{}% clears all header and footer fields
  \fancyfoot[C]{--~\thepage~--}%
  \renewcommand*{\headrulewidth}{0pt}%
  \renewcommand*{\footrulewidth}{0pt}%
}

% Shortcuts
\newcommand{\xor}{\oplus} % exclusive or
\newcommand{\true}{\textbf{T}} % true
\newcommand{\false}{\textbf{F}} % false
\newcommand{\lra}{\leftrightarrow} % iff

\newcommand{\powset}{\mathcal{P}} % power set

\newcommand{\comp}{\circ} % composition
\DeclarePairedDelimiter\ceil{\lceil}{\rceil} % ceil function
\DeclarePairedDelimiter\floor{\lfloor}{\rfloor} % floor function
\DeclarePairedDelimiter\abs{\lvert}{\rvert} % absolute value

\DeclarePairedDelimiter\paren{(}{)} % parenthesis

\newcommand{\df}{\displaystyle\frac} % displaystyle fraction
\newcommand{\qeq}{\overset{?}{=}} % questionable equality

\newcommand{\Mod}[1]{\;\mathrm{mod}\; #1} % modulo operator

% Sets
\DeclarePairedDelimiter\set{\{}{\}}
\newcommand{\unite}{\cup}
\newcommand{\inter}{\cap}

\newcommand{\reals}{\mathbb{R}}
\newcommand{\realspos}{\mathbb{R}^+} % real numbers: textbook is Z^+
\newcommand{\ints}{\mathbb{Z}}
\newcommand{\posints}{\mathbb{Z}^+}
\newcommand{\nats}{\mathbb{N}} % textbook is Z^+ and 0
\newcommand{\rats}{\mathbb{Q}}
\newcommand{\comps}{\mathbb{C}}

% Counting
\newcommand\perm[2][^n]{\prescript{#1\mkern-2.5mu}{}P_{#2}}
\newcommand\comb[2][^n]{\prescript{#1\mkern-0.5mu}{}C_{#2}}

\setlength\parindent{0pt}

% Sign Charts
\newdimen\tcolw \tcolw=2.5em % the column width
\edef\ecatcode{\catcode`&=\the\catcode`&\relax}\catcode`&=4
\def\sgchart#1#2{\vbox{\offinterlineskip\halign{\hfil##\quad&##\hfil\crcr\sgchartA#2,:,%
   \omit\sgchartR&\kern.2pt\sgchartS{.5\tcolw}\relax\sgchartE#1,\relax,%
   \sgchartS{.5\tcolw}\relax\cr
   \noalign{\kern2pt}&\def~{}\kern.5\tcolw\sgchartD#1,\relax,\cr}}}
\def\sgchartA#1:#2,{\cr\ifx,#1,\else $#1$&\sgchartB#2{}\expandafter\sgchartA\fi}
\def\sgchartB#1{\hbox to\tcolw{\hss$#1$\hss}\sgchartC}
\def\sgchartC#1{\ifx,#1,\else
   \strut\vrule\kern-.4pt\hbox to\tcolw{\hss$#1$\hss}\expandafter\sgchartC\fi}
\def\sgchartD#1#2,{\ifx\relax#1\else\hbox to\tcolw{\hss$#1#2$\hss}\expandafter\sgchartD\fi}
\def\sgchartE#1#2,{\ifx\relax#1\else
    \ifx~#1\sgchartS\tcolw\circ \else\sgchartS\tcolw\bullet\fi \expandafter\sgchartE\fi}
\def\sgchartR{\leaders\vrule height2.8pt depth-2.4pt\hfil}
\def\sgchartS#1#2{\hbox to#1{\kern-.2pt\sgchartR \ifx\relax#2\else
   \kern-.7pt$#2$\kern-.7pt\sgchartR\fi\kern-.2pt}}
\ecatcode
%++++++++++++++++++++++++++++++++++++++++

\title{
    \vspace{2in}
    \textmd{\textbf{\studyTitle}}
    \normalsize\vspace{0.1in}\\
    \vspace{0.1in}\large{\text{\class}} \\
    \vspace{0.1in}\text{\professor}\\
    \vspace{0.1in}\large\text{Midterm: \text{\examDate}}\\
    \vspace{3in}
}
\author{\name}
\date{}

\begin{document}
\maketitle
\thispagestyle{empty}
\pagebreak
\tableofcontents
\pagebreak

\section{Background}
Easy. List of topics:
\begin{enumerate}
    \item Vectors and Scalars
    \item Addition, Subtraction, and Multiplication of Vectors
    \item Dimensional Analysis
\end{enumerate}
\section{One-Dimensional Kinematics}
List of topics:
\begin{enumerate}
    \item Instantaneous Speed, Velocity, and Acceleration
    \item Average Speed, Velocity, and Acceleration
    \item Uniformly Accelerated Motion: Freely Falling Objects
\end{enumerate}
Kinematics work \textbf{only when acceleration is constant}. Take derivatives and integrals of position, velocity, and acceleration to find the equations of motion. Use the equations of motion to solve problems.
    \subsection{Instantaneous Velocity}
    \begin{equation}
        v(t) = \lim_{\Delta t \to 0} \frac{\Delta x}{\Delta t} = \frac{dx}{dt} \iff \Delta x = \int dx = \int_{t_0}^{t_1} v(t) dt
        \label{eq:instantaneous velocity}
    \end{equation}
    \subsection{Instantaneous Speed}
    \begin{equation}
        |v(t)|
    \end{equation}
    \textbf{PAY ATTENTION TO QUESTION! SPEED IS ALWAYS POSITIVE!}

    Speed is related to total distance traveled whereas velocity is related to the displacement vector.
    \begin{equation}
        \text{Displacement} \leq \text{Distance}
    \end{equation}
    \subsection{Instantaneous Acceleration}
    \begin{equation}
        a(t) = \lim_{\Delta t \to 0} \frac{\Delta v}{\Delta t} = \frac{dv}{dt} = \frac{d^2 x}{dt^2} \iff \Delta v = \int dv = \int_{t_0}^{t_1} a(t) dt
        \label{eq:instantaneous acceleration}
    \end{equation}
    \subsection{Average Velocity}
        \begin{equation}
            \overline{v} = \frac{\displaystyle \int_{t_0}^{t_1} v(t) dt}{t_1 - t_0} = \frac{\displaystyle\int_{t_0}^{t_1} dx}{t_1 - t_0} = \frac{\left[x \biggr\rvert_{t_0}^{t_1}\right]}{t_1 - t_0} = \frac{x(t_1) - x(t_0)}{t_1 - t_0} = \frac{\Delta x}{\Delta t}
            \label{eq:average velocity}
        \end{equation}
        \begin{itemize}
            \item Derived from calculus average value of a function
        \end{itemize}
        \subsection{Average Speed}
        \begin{equation}
            \text{Average Speed} = \frac{\displaystyle \int_{t_0}^{t_1} |dx|}{t_1 - t_0} = \frac{\text{total distance}}{\Delta t}
            \label{eq:average speed}
        \end{equation}
        \begin{itemize}
            \item \textbf{AVERAGE SPEED IS NOT THE ABSOLUTE VALUE OF THE AVERAGE VELOCITY!}
        \end{itemize}
        \textbf{Average Acceleration}
        \begin{equation}
            \overline{a} = \frac{\displaystyle \int_{t_0}^{t_1} a(t) dt}{t_1 - t_0} = \frac{\displaystyle\int_{t_0}^{t_1} dv}{t_1 - t_0} = \frac{\left[v \biggr\rvert_{t_0}^{t_1}\right]}{t_1 - t_0} = \frac{v(t_1) - v(t_0)}{t_1 - t_0} = \frac{\Delta v}{\Delta t}
            \label{eq:average acceleration}
        \end{equation}
        \subsection{Graphical Interpretations:} Line connecting the two points is the average velocity (Path independent). The slope of the line is the average acceleration. The absolute value area under the line is the average speed.
        \subsection{Uniformly Accelerated Motion}
        By definition:
        \begin{equation}
            a(t) = \text{constant} = a
            \label{eq:uniformly accelerated motion acceleration}
        \end{equation}
        Velocity is the integral of acceleration:
        \begin{equation}
            v(t) = \int a(t) dt = at + v_0
            \label{eq:uniformly accelerated motion velocity}
        \end{equation}
        Position is the integral of velocity:
        \begin{equation}
            x(t) = \int v(t) dt = \frac{1}{2}at^2 + v_0t + x_0
            \label{eq:uniformly accelerated motion position}
        \end{equation}
        Another unique property of UAM is that velocity is:
        \begin{equation}
            \overline{v} = \frac{v + v_0}{2}
        \end{equation}
        This can be understood by examining how the average of a linear function is the midpoint of the line.
        \subsubsection{Position Without Reference to Time}
        From \autoref{eq:uniformly accelerated motion velocity}:
        \begin{equation*}
            t = \frac{v - v_0}{a}
        \end{equation*}
        Substitute into \autoref{eq:uniformly accelerated motion position}:
        \begin{equation}
            v^2 = v_0^2 + 2a(x - x_0)
            \label{eq:position without reference to time}
        \end{equation}
    \section{Two-Dimensional Kinematics}
    List of topics:
    \begin{enumerate}
        \item Instantaneous Velocity, Speed, and Acceleration in Two Dimensions
        \item Uniformly Accelerated Motion Including Projectile Motion
        \item Relative Position, Velocity, and Acceleration
        \item Uniform Circular Motion
    \end{enumerate}
    \subsection{Position Vector}
    We can represent position in two dimensions using a vector. The position vector is defined as:
    \begin{equation}
        \vec{r} = \begin{pmatrix} x \\ y \end{pmatrix} = x\hat{i} + y\hat{j}
        \label{eq:position vector}
    \end{equation}
    \subsection{Instantaneous Velocity}
    \begin{equation}
        \vec{v}(t) \equiv \lim_{\Delta t \to 0} = \frac{\Delta \vec{r}}{\Delta t} = \frac{d\vec{r}}{dt} \iff \Delta \vec{r} = \int d\vec{r} \equiv \int_{t_0}^{t_1} \vec{v}(t) dt
        \label{eq:instantaneous velocity 2d}
    \end{equation}
    \subsection{Instantaneous Speed}
    \begin{equation}
        \abs*{\abs*{v(t)}} = \sqrt{v_x^2 + v_y^2}
        \label{eq:instantaneous speed 2d}
    \end{equation}
    \subsection{Instantaneous Acceleration}
    \begin{equation}
        \vec{a}(t) \equiv \lim_{\Delta t \to 0} \frac{\Delta \vec{v}}{\Delta t} = \frac{d\vec{v}}{dt} \equiv \frac{d^2\vec{r}}{dt^2}  \iff \Delta \vec{v} = \int d\vec{v} \equiv \int_{t_0}^{t_1} \vec{a}(t) dt
        \label{eq:instantaneous acceleration 2d}
    \end{equation}
    We can represent two-dimensional vectors by sets of one-dimensional vectors. For example, the velocity vector can be represented by two one-dimensional vectors:
    \begin{align}
        \vec{v}(t) &= \begin{pmatrix} v_x(t) \\ v_y(t) \end{pmatrix} = v_x(t)\hat{i} + v_y(t)\hat{j} \nonumber \\
        \frac{d\vec{r}}{dt} &= \begin{pmatrix} \displaystyle\frac{dx}{dt} \\[.75em] \displaystyle\frac{dy}{dt} \end{pmatrix} = \frac{dx}{dt}\hat{i} + \frac{dy}{dt}\hat{j}
        \label{eq:velocity vector components}
    \end{align}
    The two dimensions are independent of each other (other than being synchronized by time). We can create a parametric equation for the velocity vector:
    \begin{equation}
        \begin{cases}
            v_x(t) = \displaystyle\frac{dx}{dt} = v_{x_0} + a_xt \\[.75em]
            v_y(t) = \displaystyle\frac{dy}{dt} = v_{y_0} + a_yt
        \end{cases}
        \label{eq:velocity vector parametric}
    \end{equation}
    \subsection{Displacement Vector}
    Displacement vector points from the initial position to the final position. It is defined as:
    \begin{equation}
        \Delta \vec{r} = \begin{pmatrix} \Delta x \\ \Delta y \end{pmatrix} = \Delta x\hat{i} + \Delta y\hat{j}
        \label{eq:displacement vector}
    \end{equation}
    \subsection{Average Velocity}
    The average velocity during a given time interval is parallel to the displacement vector (Multiplying by a scalar does not change the direction of a vector):
    \begin{equation}
        \overline{\vec{v}} = \frac{\Delta \vec{r}}{\Delta t} 
        \label{eq:average velocity 2d}
    \end{equation}
    \subsection{Projectile Motion}
    Conversion from rectangular to polar form:
    \begin{align}
        v &= \sqrt{v_x^2 + v_y^2} \\
        \theta &= \tan^{-1}\paren*{\frac{v_y}{v_x}}
        \label{rectangular to polar}
    \end{align}
    Conversion from polar to rectangular form:
    \begin{align}
        v_x &= v\cos\theta \\
        v_y &= v\sin\theta
        \label{polar to rectangular}
    \end{align}
    We can separate projectile motion into two components: horizontal and vertical. They are independent from one another.
    \subsubsection{Horizontal Motion}
    Horizontal motion is uniform motion in the $x$-direction. The horizontal velocity is constant. The horizontal acceleration is zero.
    \begin{align}
        v_x(t) &= v_{x_0} = \text{constant} \\
        x(t) &=x_0 + v_{x_0}t
    \end{align}
    \subsubsection{Vertical Motion}
    The acceleration in the $y$-direction is constant. The initial velocity in the $y$-direction is zero. The initial position in the $y$-direction is zero.
    \begin{align}
        a_y &= -g \\
        v_y(t) &= v_{y_0} + a_yt \\
        y(t) &= \frac{1}{2}a_yt^2 + v_{y_0}t + y_0 \\
        v_y^2 &= v_{y_0}^2 - 2g(y - y_0)
    \end{align}
\end{document}