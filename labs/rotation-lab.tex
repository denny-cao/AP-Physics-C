%++++++++++++++++++++++++++++++++++++++++
\documentclass[article, 11pt]{article}
\usepackage{float}
\usepackage{setspace}
\usepackage{tabu} % extra features for tabular environment
\usepackage{amsmath}  % improve math presentation
\usepackage{graphicx} % takes care of graphic including machinery
\usepackage[margin=1in]{geometry} % decreases margins
\usepackage{cite} % takes care of citations
\usepackage[final]{hyperref} % adds hyper links inside the generated pdf file
\usepackage{tikz}
\usepackage{caption} 
\usepackage{fancyhdr}
\usepackage{amssymb} % symbols like /therefore
\usepackage{amsthm} % proofs
\usepackage{enumerate} % lettered lists
\usepackage{mathtools} % macros
\usetikzlibrary{scopes}
\usepackage{xcolor} \pagecolor[rgb]{0.12549019607,0.1294117647,0.13725490196} \color[rgb]{0.82352941176,0.76862745098,0.62745098039} % dark theme
\hypersetup{
	colorlinks=true,       % false: boxed links; true: colored links
	linkcolor=blue,        % color of internal links
	citecolor=blue,        % color of links to bibliography
	filecolor=magenta,     % color of file links
	urlcolor=blue         
}
\usepackage{titling}
\renewcommand\maketitlehooka{\null\mbox{}\vfill}
\renewcommand\maketitlehookd{\vfill\null}
\usepackage{siunitx} % units
\usepackage{verbatim} 
\newcommand{\labTitle}{Spinning Things Lab: The Windmill and I}
\newcommand{\class}{AP Physics C}
\newcommand{\professor}{Mr. Perkins}
\newcommand{\name}{Denny Cao}
\pagestyle{fancy}
\fancyhf{}% clears all header and footer fields
\fancyfoot[C]{--~\thepage~--}
\renewcommand*{\headrulewidth}{0.4pt}
\renewcommand*{\footrulewidth}{0pt}
\lhead{\name}
\chead{\class: \labTitle}
\rhead{\professor}


\fancypagestyle{plain}{%
  \fancyhf{}% clears all header and footer fields
  \fancyfoot[C]{--~\thepage~--}%
  \renewcommand*{\headrulewidth}{0pt}%
  \renewcommand*{\footrulewidth}{0pt}%
}

% Shortcuts
\DeclarePairedDelimiter\ceil{\lceil}{\rceil} % ceil function
\DeclarePairedDelimiter\floor{\lfloor}{\rfloor} % floor function

\DeclarePairedDelimiter\paren{(}{)} % parenthesis

\newcommand{\df}{\displaystyle\frac} % displaystyle fraction
\newcommand{\qeq}{\overset{?}{=}} % questionable equality

\newcommand{\Mod}[1]{\;\mathrm{mod}\; #1} % modulo operator

% Sets
\DeclarePairedDelimiter\set{\{}{\}}
\newcommand{\unite}{\cup}
\newcommand{\inter}{\cap}

\newcommand{\reals}{\mathbb{R}} % real numbers: textbook is Z^+ and 0
\newcommand{\ints}{\mathbb{Z}}
\newcommand{\nats}{\mathbb{N}}
\newcommand{\rats}{\mathbb{Q}}


% Counting
\newcommand\perm[2][^n]{\prescript{#1\mkern-2.5mu}{}P_{#2}}
\newcommand\comb[2][^n]{\prescript{#1\mkern-0.5mu}{}C_{#2}}

\setlength\parindent{0pt}

% Sign Charts
\newdimen\tcolw \tcolw=2.5em % the column width
\edef\ecatcode{\catcode`&=\the\catcode`&\relax}\catcode`&=4
\def\sgchart#1#2{\vbox{\offinterlineskip\halign{\hfil##\quad&##\hfil\crcr\sgchartA#2,:,%
   \omit\sgchartR&\kern.2pt\sgchartS{.5\tcolw}\relax\sgchartE#1,\relax,%
   \sgchartS{.5\tcolw}\relax\cr
   \noalign{\kern2pt}&\def~{}\kern.5\tcolw\sgchartD#1,\relax,\cr}}}
\def\sgchartA#1:#2,{\cr\ifx,#1,\else $#1$&\sgchartB#2{}\expandafter\sgchartA\fi}
\def\sgchartB#1{\hbox to\tcolw{\hss$#1$\hss}\sgchartC}
\def\sgchartC#1{\ifx,#1,\else
   \strut\vrule\kern-.4pt\hbox to\tcolw{\hss$#1$\hss}\expandafter\sgchartC\fi}
\def\sgchartD#1#2,{\ifx\relax#1\else\hbox to\tcolw{\hss$#1#2$\hss}\expandafter\sgchartD\fi}
\def\sgchartE#1#2,{\ifx\relax#1\else
    \ifx~#1\sgchartS\tcolw\circ \else\sgchartS\tcolw\bullet\fi \expandafter\sgchartE\fi}
\def\sgchartR{\leaders\vrule height2.8pt depth-2.4pt\hfil}
\def\sgchartS#1#2{\hbox to#1{\kern-.2pt\sgchartR \ifx\relax#2\else
   \kern-.7pt$#2$\kern-.7pt\sgchartR\fi\kern-.2pt}}
\ecatcode
%++++++++++++++++++++++++++++++++++++++++
\title{
    \vspace{2in}
    \textmd{\textbf{\labTitle}}
    \normalsize\vspace{0.1in}\\
    \vspace{0.1in}\large{\text{\class: \professor}}
    \vspace{3in}
}

\author{\name}
\date{\today}

\begin{document}
    \maketitle
    \thispagestyle{empty}
    \pagebreak
    
    \section{Observed Acceleration}
    \subsection{Procedure}
    \noindent 
    Using a windmill with 4 rods with weights at their ends, we wounded a string on the rung of the windmill, attaching a 100g to the end of the string. We then dropped the weight 0.77m and measured the time it took to reach the bottom. After the weight dropped, we measured the time it took for the windmill to make 10 revolutions to compute the angular velocity of the windmill. With this, we were able to compute the angular acceleration and moment of inertia of the windmill.

    \subsection{Data}
    \begin{figure}[H]
        \begin{center}
            \begin{tabular}{|c|c|c|}
                \hline
                \textbf{Measurement} & \textbf{Variable} & \textbf{Value} \\
                \hline
                Distance & $h$ & \SI{0.77}{\meter} \\
                Mass of dropped weight & $m$ & \SI{0.1}{\kilogram} \\
                Time to drop & $t_d$ & \SI{20.08}{\second} \\
                Time to make 10 revolutions & $t_r$ & \SI{16.9}{\second} \\
                \hline
            \end{tabular}
        \end{center}
        \caption{Recorded Data}
    \end{figure}

    \subsection{Analysis}
    \textbf{Angular Velocity After Weight Dropped:}
    \begin{align*}
        \omega   &= \frac{\theta}{t}  \\
        \omega_f &= \frac{2\pi(10)}{t_r} = \frac{20\pi}{16.9} \\
                 &\approx \SI{3.7179}{\frac{\radian}{\second}}
    \end{align*}
    \textbf{Angular Acceleration:}
    \begin{align*}
        \alpha &= \frac{\omega_f - \omega_i}{t} \\
               &= \frac{3.7179 - 0}{20.08} \\
               &\approx \SI{0.1852}{\frac{\radian}{\second^2}}
    \end{align*}
    \textbf{Moment of Inertia:}
    \begin{align*}
        mgh &= \frac{1}{2}I{\omega_f}^2 \\
        I   &= \frac{2mgh}{{\omega_f}^2} \\
            &= \frac{2(0.1)(9.81)(0.77)}{(3.7179)^2} \\
            &\approx \SI{0.10929}{\kilogram\meter^2}
    \end{align*}

    \section{Theoretical Acceleration}
    \subsection{Procedure}
    \noindent
    We recorded the masses of the rods, weights, and the center of the windmill as well as the lengths of the rods. Since the rod is inserted into the pivot of the windmill but does not go to the center, we measured the distance from the pivot to the rod to compute the moment of inertia. 

    \subsection{Data}
    \begin{figure}[H]
        \begin{center}
            \begin{tabular}{|c|c|c|}
                \hline
                \textbf{Measurement} & \textbf{Variable} & \textbf{Value} \\
                \hline
                Mass of rod & $m_r$ & \SI{0.074}{\kilogram} \\
                Mass of weight & $m_w$ & \SI{0.186}{\kilogram} \\
                Radius of weight & $r_w$ & \SI{0.34}{\meter} \\
                Length of rod & $l$ & \SI{0.3}{\meter} \\
                Distance from center to rod & $d_r$ & \SI{0.05}{\meter} \\
                Distance from center to weight & $d_w$ & \SI{0.33}{\meter} \\
                Moment of inertia of pulley assembly alone & $I_p$ & \SI{0.00058}{\kilogram\meter^2} \\
                \hline
            \end{tabular}
        \end{center}
        \caption{Recorded Data}
    \end{figure}
    \subsection{Analysis}
    \begin{comment}

    The moment of inertia for the windmill is given by:
    \begin{equation*}
        I = I_{\text{pulley}} + 4I_{\text{rod}} + 4I_{\text{weight}} + 4I_{\text{parallel axis contribution}}
    \end{equation*}
    Where $I_p$ is the moment of inertia of the pulley assembly alone, $I_r$ is the moment of inertia of one rod, and $I_w$ is the moment of inertia of one weight. The Parallel Axis Theorem can be used to find the moment of inertia of an object, $o$ about an axis parallel to the axis of rotation:
    \begin{equation*}
        I_o = I_{cm} + Md^2 
    \end{equation*}
    where $I_{cm}$ is the moment of inertia of the object about its center of mass, $M$ is the mass of the object, and $d$ is the distance from the center of mass to the axis of rotation. \\
    
    We can use this to find the moment of inertia of the rods and weights: \\

    \textbf{Moment of Inertia of Rod $I_r$:}
    \begin{align*}
        I_r &= I_{r_{cm}} + m_r(d_r)^2 \\
        \intertext{The moment of inertia of a rod about its end is given by: $I = \df{1}{3}ml^2$}
            &= \frac{1}{3}m_r l^2 + m_r(d_r)^2 \\
            &= \frac{1}{3}(0.074)(0.3)^2 + (0.074)(0.05)^2 \\
            &\approx \SI{0.0024}{\kilogram\meter^2}
    \end{align*}
    \textbf{Moment of Inertia of Weight $I_w$:}
    \begin{align*}
        I_w &= I_{w_{cm}} + m_w(d_w)^2 \\
    \intertext{The moment of inertia of a point mass is given by: $I = mr^2$} 
            &= m_w(r_w)^2 + m_w(d_w)^2 \\
            &= (0.186)(0.1)^2 + (0.186)(0.33)^2 \\
            &\approx \SI{0.0221}{\kilogram\meter^2}
    \end{align*}
    \textbf{Moment of Inertia of Windmill:}
    \begin{align*}
        I &= I_p + 4I_r + 4I_w \\
          &= 0.00058 + 4(0.0024) + 4(0.0221) \\
          &= \SI{0.9858}{\kilogram\meter^2}
    \end{align*}
    \end{comment}
    \begin{comment}
    \textbf{Moment of Inertia of Rod $I_\text{rod}$:}
    \begin{align*}
        I_r &= \int r^2 dm \\
            &= \frac{m_r}{l} \int_{d_r}^{l + d_r} {x^2 dx} \\
            &= \frac{0.074}{0.3} \int_{0.05}^{0.35} {x^2 dx} \\
            &\approx 0.003515
    \end{align*}
    \textbf{Moment of Inertia of Weight $I_\text{weight}$:}
    \begin{align*}
    \intertext{The inertia of a point mass is given by $I = mr^2$}
        I_w &= m_w(r_w)^2 \\
            &= (0.186)(0.34)^2 \\
            &\approx 0.0215
    \end{align*}
    \textbf{Parallel Axis Contribution $I_\text{parallel axis contribution}$:}
    \begin{align*}
        I_\text{pac} &= m_w(d_w)^2 \\
            &= (0.186)(0.33)^2 \\
            &\approx 0.0203
    \end{align*}
    \textbf{Moment of Inertia of Windmill:}
    \begin{align*}
        I &= I_{\text{pulley}} + 4I_{\text{rod}} + 4I_{\text{weight}} + 4I_{\text{parallel axis contribution}} \\
            &= 0.00058 + 4(0.003515) + 4(0.0215) + 4(0.0203) \\
            &\approx 0.18184
    \end{align*}
    \end{comment}
    
    The moment of inertia of the windmill is given by:
    \begin{equation*}
        I = I_p + 4I_r + 4I_w 
    \end{equation*}
    \textbf{Inertia of Rod $I_r$:} 
    \begin{align*}
        I_r &= \int r^2 dm \\
            &= \frac{m_r}{l} \int_{d_r}^{l + d_r} {x^2 dx} \\
            &= \frac{0.074}{0.3} \int_{0.05}^{0.35} {x^2 dx} \\
            &\approx \SI{0.00352}{\kilogram\meter^2}
    \end{align*}
    \textbf{Inertia of Weight $I_w$:}
    \begin{align*}
        I_w &= m_w(r_w)^2 \\
            &= (0.186)(0.34)^2 \\
            &\approx \SI{0.02150}{\kilogram\meter^2}
    \end{align*}
    \textbf{Inertia of Windmill:}
    \begin{align*}
        I &= I_p + 4I_r + 4I_w \\
            &= 0.00058 + 4(0.00352) + 4(0.02150) \\
            &\approx \SI{0.10066}{\kilogram\meter^2}
    \end{align*}
    \section{Conclusion}
    \noindent
    The percent error in the moment of inertia of the windmill is given by:
    \begin{align*}
        \delta &= \frac{|I_A - I_E|}{I_E} \cdot 100\% \\
               &= \frac{|0.10929 - 0.10066|}{0.10066} \cdot 100\% \\
               &= 8.57\%
    \end{align*}
    This is a relatively small error that can be attributed to the fact that the theoretical calculations disregarded non-conservative forces such as friction and air resistance. This is also why it is reasonable that $I_A > I_E$. Friction and air resistance cause the windmill to have a decrease in angular velocity, and thus an increase in moment of inertia from the equation $I = \df{1}{2}I\omega^2$. We will compute the energy lost.
    \begin{comment}
    \begin{align*}
        E_E                      &= E_A + W_\text{lost} \\
        \frac{1}{2}I_E\omega_E^2 &= \frac{1}{2}I_A\omega_A^2 + W_\text{lost} 
    \end{align*}
    We first compute $\omega_E$, the angular velocity of the expected angular velocity:
    \begin{align*}
        mgh      &= \frac{1}{2}I_E\omega_E^2 \\
        \omega_E &= \sqrt{\frac{2mgh}{I_E}} \\
                 &= \sqrt{\frac{2(0.1)(9.81)(0.77)}{0.10066}} \\
                 &\approx \SI{3.8741}{\frac{\radian}{\second}}
    \end{align*}
    \end{comment}
    \begin{align*}
        \tau &= F \times r \\
        \tau &= I \alpha \\
        F \times r &= I \alpha \\
        \alpha &= \frac{F \times r}{I}
    \end{align*}
    Let $F$ be the tension force and $r$ be the distance from the center of the pulley to the string. We first solve for $F$.
    \begin{equation*}
        F = F_W - F_\text{net}
    \end{equation*}
    \begin{align*}
        F_\text{net} &= ma \\
        a &= \frac{2h}{{t_d}^2} \\
        F_\text{net} &= m\frac{2h}{{t_d}^2} \\
        F_W &= mg
    \end{align*}
    \begin{align*}
        F &= mg - m\frac{2h}{{t_d}^2} \\
          &= (0.1)(9.81) - (0.1)\frac{2(0.77)}{20.08^2} \\
          &\approx \SI{0.9806}{\newton}
    \end{align*}
    We can compute the angular acceleration of the windmill using:
    \begin{alignat*}{2}
        \begin{aligned}
            \alpha_A  &= \frac{F \times r}{I_A} \\
                      &= \frac{0.9806(0.02)}{0.10929} \\
                      &\approx \SI{0.1794}{\frac{\radian}{\second^2}}
        \end{aligned} \quad &
        \begin{aligned}
            \alpha_E &= \frac{F \times r}{I_E} \\
                     &= \frac{0.9806(0.02)}{0.10066} \\
                     &\approx \SI{0.1948}{\frac{\radian}{\second^2}}
        \end{aligned}
    \end{alignat*}
    We can compute the final angular velocity of the windmill using:
    \begin{equation*}
        \omega = \omega_0 + \alpha t
    \end{equation*}
    \begin{alignat*}{2}
        \begin{aligned}
            \omega_A &= 0.1794(20.08) \\
                     &\approx \SI{3.6024}{\frac{\radian}{\second}}
        \end{aligned} \quad &
        \begin{aligned}0.1948(20.08) \\
                     &= \SI{3.9116}{\frac{\radian}{\second}}
        \end{aligned}
    \end{alignat*}
\end{document}

\begin{comment}
mgh = 1/2 iw^2
since i down, w^2 up 
we are finding what causes w^2 to slow down
energy in system is same. both equal mgh. 

1/2 iw^2 + wlost = mgh 
in the case of theoreical, wlost = 0. therefore, 1/2 iw^2 = mgh. 
but what about in the case of actual? we directly solve 1/2iw^2. So what
we essentially have is:

1/2 i_e w_e^2 = mgh
1/2 i_a w_a^2 = mgh
we can compare torques.
torque = r x f
r x f = I alpha
r x f will be the same for both.

what can we do with acceleration. 
backsolve for w_e^2 and w_a^2.
then we can compare energies. 
\end{comment}