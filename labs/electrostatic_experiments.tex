%++++++++++++++++++++++++++++++++++++++++
\documentclass[article, 12pt]{article}
\usepackage{float}
\usepackage{setspace}
\usepackage{tabu} % extra features for tabular environment
\usepackage{amsmath}  % improve math presentation
\usepackage{graphicx} % takes care of graphic including machinery
\usepackage[margin=1in]{geometry} % decreases margins
\usepackage{cite} % takes care of citations
\usepackage[final]{hyperref} % adds hyper links inside the generated pdf file
\usepackage{tikz}
\usepackage{caption} 
\usepackage{fancyhdr}
\usepackage{amssymb} % symbols like /therefore
\usepackage{amsthm} % proofs
\usepackage{enumerate} % lettered lists
\usepackage{mathtools} % macros
\usepackage{multirow} % multirow tables
\usetikzlibrary{scopes}
% \usepackage{xcolor} \pagecolor[rgb]{0.12549019607,0.1294117647,0.13725490196} \color[rgb]{0.82352941176,0.76862745098,0.62745098039} % dark theme
\hypersetup{
	colorlinks=true,       % false: boxed links; true: colored links
	linkcolor=blue,        % color of internal links
	citecolor=blue,        % color of links to bibliography
	filecolor=magenta,     % color of file links
	urlcolor=blue         
}
\usepackage{physics}
\usepackage{siunitx}
\usepackage{tikz,pgfplots}
\usepackage[outline]{contour} % glow around text
\usetikzlibrary{calc}
\usetikzlibrary{angles,quotes} % for pic
\usetikzlibrary{arrows.meta}
\tikzset{>=latex} % for LaTeX arrow head
\contourlength{1.2pt}

\colorlet{xcol}{blue!70!black}
\colorlet{vcol}{green!60!black}
\colorlet{myred}{red!70!black}
\colorlet{myblue}{blue!70!black}
\colorlet{mygreen}{green!70!black}
\colorlet{mydarkred}{myred!70!black}
\colorlet{mydarkblue}{myblue!60!black}
\colorlet{mydarkgreen}{mygreen!60!black}
\colorlet{acol}{red!50!blue!80!black!80}
\tikzstyle{CM}=[red!40!black,fill=red!80!black!80]
\tikzstyle{xline}=[xcol,thick,smooth]
\tikzstyle{mass}=[line width=0.6,red!30!black,fill=red!40!black!10,rounded corners=1,
                  top color=red!40!black!20,bottom color=red!40!black!10,shading angle=20]
\tikzstyle{faded mass}=[dashed,line width=0.1,red!30!black!40,fill=red!40!black!10,rounded corners=1,
                        top color=red!40!black!10,bottom color=red!40!black!10,shading angle=20]
\tikzstyle{rope}=[brown!70!black,very thick,line cap=round]
\def\rope#1{ \draw[black,line width=1.4] #1; \draw[rope,line width=1.1] #1; }
\tikzstyle{force}=[->,myred,very thick,line cap=round]
\tikzstyle{velocity}=[->,vcol,very thick,line cap=round]
\tikzstyle{Fproj}=[force,myred!40]
\tikzstyle{myarr}=[-{Latex[length=3,width=2]},thin]
\def\tick#1#2{\draw[thick] (#1)++(#2:0.12) --++ (#2-180:0.24)}
\DeclareMathOperator{\sn}{sn}
\DeclareMathOperator{\cn}{cn}
\DeclareMathOperator{\dn}{dn}
\def\N{80} % number of samples in plots


\usepackage{titling}
\renewcommand\maketitlehooka{\null\mbox{}\vfill}
\renewcommand\maketitlehookd{\vfill\null}
\usepackage{siunitx} % units
\usepackage{verbatim} 
\newcommand{\labTitle}{Electrostatic Experiments}
\newcommand{\class}{AP Physics C}
\newcommand{\professor}{Mr. Perkins}
\newcommand{\name}{Denny Cao}
\pagestyle{fancy}
\fancyhf{}% clears all header and footer fields
\fancyfoot[C]{--~\thepage~--}
\renewcommand*{\headrulewidth}{0.4pt}
\renewcommand*{\footrulewidth}{0pt}
\lhead{\name}
\chead{\class: \labTitle}
\rhead{\professor}


\fancypagestyle{plain}{%
  \fancyhf{}% clears all header and footer fields
  \fancyfoot[C]{--~\thepage~--}%
  \renewcommand*{\headrulewidth}{0pt}%
  \renewcommand*{\footrulewidth}{0pt}%
}

% Shortcuts
\DeclarePairedDelimiter\ceil{\lceil}{\rceil} % ceil function
\DeclarePairedDelimiter\floor{\lfloor}{\rfloor} % floor function

\DeclarePairedDelimiter\paren{(}{)} % parenthesis

\newcommand{\df}{\displaystyle\frac} % displaystyle fraction
\newcommand{\qeq}{\overset{?}{=}} % questionable equality

\newcommand{\Mod}[1]{\;\mathrm{mod}\; #1} % modulo operator

% Sets
\DeclarePairedDelimiter\set{\{}{\}}
\newcommand{\unite}{\cup}
\newcommand{\inter}{\cap}

\newcommand{\reals}{\mathbb{R}} % real numbers: textbook is Z^+ and 0
\newcommand{\ints}{\mathbb{Z}}
\newcommand{\nats}{\mathbb{N}}
\newcommand{\rats}{\mathbb{Q}}

\newcommand{\degree}{^\circ}
\newcommand{\tplint}[6]{\int\limits^{#1}_{#2}\int\limits^{#3}_{#4}\int\limits^{#5}_{#6}} % triple integral bounds
% Counting
\newcommand{\dblint}[4]{\int\limits^{#1}_{#2}\int\limits^{#3}_{#4}} % double integral bounds
\newcommand{\sglint}[2]{\int\limits^{#1}_{#2}} % integral bounds
\newcommand\perm[2][^n]{\prescript{#1\mkern-2.5mu}{}P_{#2}}
\newcommand\comb[2][^n]{\prescript{#1\mkern-0.5mu}{}C_{#2}}

\setlength\parindent{0pt}

% Sign Charts
\newdimen\tcolw \tcolw=2.5em % the column width
\edef\ecatcode{\catcode`&=\the\catcode`&\relax}\catcode`&=4
\def\sgchart#1#2{\vbox{\offinterlineskip\halign{\hfil##\quad&##\hfil\crcr\sgchartA#2,:,%
   \omit\sgchartR&\kern.2pt\sgchartS{.5\tcolw}\relax\sgchartE#1,\relax,%
   \sgchartS{.5\tcolw}\relax\cr
   \noalign{\kern2pt}&\def~{}\kern.5\tcolw\sgchartD#1,\relax,\cr}}}
\def\sgchartA#1:#2,{\cr\ifx,#1,\else $#1$&\sgchartB#2{}\expandafter\sgchartA\fi}
\def\sgchartB#1{\hbox to\tcolw{\hss$#1$\hss}\sgchartC}
\def\sgchartC#1{\ifx,#1,\else
   \strut\vrule\kern-.4pt\hbox to\tcolw{\hss$#1$\hss}\expandafter\sgchartC\fi}
\def\sgchartD#1#2,{\ifx\relax#1\else\hbox to\tcolw{\hss$#1#2$\hss}\expandafter\sgchartD\fi}
\def\sgchartE#1#2,{\ifx\relax#1\else
    \ifx~#1\sgchartS\tcolw\circ \else\sgchartS\tcolw\bullet\fi \expandafter\sgchartE\fi}
\def\sgchartR{\leaders\vrule height2.8pt depth-2.4pt\hfil}
\def\sgchartS#1#2{\hbox to#1{\kern-.2pt\sgchartR \ifx\relax#2\else
   \kern-.7pt$#2$\kern-.7pt\sgchartR\fi\kern-.2pt}}
\ecatcode
%++++++++++++++++++++++++++++++++++++++++
\title{
    \vspace{2in}
    \textmd{\textbf{\labTitle}}
    \normalsize\vspace{0.1in}\\
    \vspace{0.1in}\large{\text{\class: \professor}}
    \vspace{3in}
}

\author{\name}
\date{February, 8 2023}

\begin{document}
    \maketitle
    \thispagestyle{empty}
    \pagebreak
    \section*{Experience 7: Interactions of U and L Tapes with Other Objects}
    \subsection*{Introduction}
    We investigated the behavior of neutral objects and charged objects. We predict that the U-tape and L-tape will attract to each other, and that the U-tape and L-tape will attract to the neutral objects.

    \subsection*{Procedure}
    We used scissors, a wooden rod, and a Styrofoam cup---these three objects have a neutral charge. We used three pieces of tape on top of each other---a base piece, an ``L-tape'' (middle), and a ``U-tape'' (upper). As the U-tape and L-tape attracted each other, we knew that they were oppositely charged or one was neutral and one was charged. We moved each tape closer to one of the neutral objects to test if the two tapes were charged. If the tapes did not attract to the object, then they were neutral. If they did, they were charged.

    \subsection*{Conclusion}
    We found that the U-tape and L-tape were charged, as they attracted to the neutral objects. This is because, if the tapes were charged, it would cause the neutral object to become polarized, with the opposite charge of the tape moving closer to the tape and the same charge of the tape moving away from the tape. The force of attraction is greater than the force of repulsion, as the distance between different charges is less than the distance between the same charges. The U-tape and L-tape were oppositely charged, as they attracted to each other.

    \section*{Experiment 10: Interaction Through a Piece of Paper}
    \subsection*{Introduction}
    We investigated whether or not an electrostatic interaction would occur even with an object in between---in this case, a thin piece of paper. We predict that the two tapes will still attract to each other.
    
    \subsection*{Procedure}
    Using tapes that were charged to the same charge by taping them on a table, we held both tapes on opposit sides of the paper. We then moved the tapes closer to each other.

    \subsection*{Conclusion}
    The two pieces of tape still experienced a repulsion force, despite the paper in between the two. Paper is not a good insulator, allowing the charges to still interact with each other. It is still possible for the charges to interact with each other, even if there is an object in between them.

    \section*{Experiment 11: Is Tape a Conductor or Insulator?}
    \subsection*{Introduction}
    We investigated whether or not tape is a conductor or insulator. We predict that tape is an insulator, as it is made of plastic and is not a good conductor of electricity.

    \subsection*{Procedure}
    We hung a tape, with the top half being charged and the bottom have being neutral. This was done by taping the two pieces of tape on top of one another but leaving half of the top tape hanging off and sticking them onto a table. After pulling them apart, the top tape would be half charged and half neutral. We then investigated what would occur when a neutral object---a pair of scissors---was moved close to the tape.

    \subsection*{Conclusion}
When we moved the scissors towards the top half of the tape, it attracted. When we moved the scissors towards the bottom half, nothing occurred. We conclude from this that tape is more of an insulator than a conductor. This is because, if tape was a conductor, then the charge would have evenly spread across the piece of tape, making the entire tape charged, meaning the scissors would have been attracted regardless if moving towards the top or the bottom.
\end{document}
