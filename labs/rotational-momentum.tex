%++++++++++++++++++++++++++++++++++++++++
\documentclass[article, 12pt]{article}
\usepackage{float}
\usepackage{setspace}
\usepackage{tabu} % extra features for tabular environment
\usepackage{amsmath}  % improve math presentation
\usepackage{graphicx} % takes care of graphic including machinery
\usepackage[margin=1in]{geometry} % decreases margins
\usepackage{cite} % takes care of citations
\usepackage[final]{hyperref} % adds hyper links inside the generated pdf file
\usepackage{tikz}
\usepackage{caption} 
\usepackage{fancyhdr}
\usepackage{amssymb} % symbols like /therefore
\usepackage{amsthm} % proofs
\usepackage{enumerate} % lettered lists
\usepackage{mathtools} % macros
\usepackage{multirow} % multirow tables
\usetikzlibrary{scopes}
% \usepackage{xcolor} \pagecolor[rgb]{0.12549019607,0.1294117647,0.13725490196} \color[rgb]{0.82352941176,0.76862745098,0.62745098039} % dark theme
\hypersetup{
	colorlinks=true,       % false: boxed links; true: colored links
	linkcolor=blue,        % color of internal links
	citecolor=blue,        % color of links to bibliography
	filecolor=magenta,     % color of file links
	urlcolor=blue         
}
\usepackage{physics}
\usepackage{siunitx}
\usepackage{tikz,pgfplots}
\usepackage[outline]{contour} % glow around text
\usetikzlibrary{calc}
\usetikzlibrary{angles,quotes} % for pic
\usetikzlibrary{arrows.meta}
\tikzset{>=latex} % for LaTeX arrow head
\contourlength{1.2pt}

\colorlet{xcol}{blue!70!black}
\colorlet{vcol}{green!60!black}
\colorlet{myred}{red!70!black}
\colorlet{myblue}{blue!70!black}
\colorlet{mygreen}{green!70!black}
\colorlet{mydarkred}{myred!70!black}
\colorlet{mydarkblue}{myblue!60!black}
\colorlet{mydarkgreen}{mygreen!60!black}
\colorlet{acol}{red!50!blue!80!black!80}
\tikzstyle{CM}=[red!40!black,fill=red!80!black!80]
\tikzstyle{xline}=[xcol,thick,smooth]
\tikzstyle{mass}=[line width=0.6,red!30!black,fill=red!40!black!10,rounded corners=1,
                  top color=red!40!black!20,bottom color=red!40!black!10,shading angle=20]
\tikzstyle{faded mass}=[dashed,line width=0.1,red!30!black!40,fill=red!40!black!10,rounded corners=1,
                        top color=red!40!black!10,bottom color=red!40!black!10,shading angle=20]
\tikzstyle{rope}=[brown!70!black,very thick,line cap=round]
\def\rope#1{ \draw[black,line width=1.4] #1; \draw[rope,line width=1.1] #1; }
\tikzstyle{force}=[->,myred,very thick,line cap=round]
\tikzstyle{velocity}=[->,vcol,very thick,line cap=round]
\tikzstyle{Fproj}=[force,myred!40]
\tikzstyle{myarr}=[-{Latex[length=3,width=2]},thin]
\def\tick#1#2{\draw[thick] (#1)++(#2:0.12) --++ (#2-180:0.24)}
\DeclareMathOperator{\sn}{sn}
\DeclareMathOperator{\cn}{cn}
\DeclareMathOperator{\dn}{dn}
\def\N{80} % number of samples in plots


\usepackage{titling}
\renewcommand\maketitlehooka{\null\mbox{}\vfill}
\renewcommand\maketitlehookd{\vfill\null}
\usepackage{siunitx} % units
\usepackage{verbatim} 
\newcommand{\labTitle}{Rotational Momentum}
\newcommand{\class}{AP Physics C}
\newcommand{\professor}{Mr. Perkins}
\newcommand{\name}{Denny Cao}
\newcommand{\dueDate}{January 11, 2023}
\pagestyle{fancy}
\fancyhf{}% clears all header and footer fields
\fancyfoot[C]{--~\thepage~--}
\renewcommand*{\headrulewidth}{0.4pt}
\renewcommand*{\footrulewidth}{0pt}
\lhead{\name}
\chead{\class: \labTitle}
\rhead{\professor}


\fancypagestyle{plain}{%
  \fancyhf{}% clears all header and footer fields
  \fancyfoot[C]{--~\thepage~--}%
  \renewcommand*{\headrulewidth}{0pt}%
  \renewcommand*{\footrulewidth}{0pt}%
}

% Shortcuts
\DeclarePairedDelimiter\ceil{\lceil}{\rceil} % ceil function
\DeclarePairedDelimiter\floor{\lfloor}{\rfloor} % floor function

\DeclarePairedDelimiter\paren{(}{)} % parenthesis

\newcommand{\df}{\displaystyle\frac} % displaystyle fraction
\newcommand{\qeq}{\overset{?}{=}} % questionable equality

\newcommand{\Mod}[1]{\;\mathrm{mod}\; #1} % modulo operator

% Sets
\DeclarePairedDelimiter\set{\{}{\}}
\newcommand{\unite}{\cup}
\newcommand{\inter}{\cap}

\newcommand{\reals}{\mathbb{R}} % real numbers: textbook is Z^+ and 0
\newcommand{\ints}{\mathbb{Z}}
\newcommand{\nats}{\mathbb{N}}
\newcommand{\rats}{\mathbb{Q}}

\newcommand{\degree}{^\circ}
\newcommand{\tplint}[6]{\int\limits^{#1}_{#2}\int\limits^{#3}_{#4}\int\limits^{#5}_{#6}} % triple integral bounds
% Counting
\newcommand{\dblint}[4]{\int\limits^{#1}_{#2}\int\limits^{#3}_{#4}} % double integral bounds
\newcommand{\sglint}[2]{\int\limits^{#1}_{#2}} % integral bounds
\newcommand\perm[2][^n]{\prescript{#1\mkern-2.5mu}{}P_{#2}}
\newcommand\comb[2][^n]{\prescript{#1\mkern-0.5mu}{}C_{#2}}

\setlength\parindent{0pt}

% Sign Charts
\newdimen\tcolw \tcolw=2.5em % the column width
\edef\ecatcode{\catcode`&=\the\catcode`&\relax}\catcode`&=4
\def\sgchart#1#2{\vbox{\offinterlineskip\halign{\hfil##\quad&##\hfil\crcr\sgchartA#2,:,%
   \omit\sgchartR&\kern.2pt\sgchartS{.5\tcolw}\relax\sgchartE#1,\relax,%
   \sgchartS{.5\tcolw}\relax\cr
   \noalign{\kern2pt}&\def~{}\kern.5\tcolw\sgchartD#1,\relax,\cr}}}
\def\sgchartA#1:#2,{\cr\ifx,#1,\else $#1$&\sgchartB#2{}\expandafter\sgchartA\fi}
\def\sgchartB#1{\hbox to\tcolw{\hss$#1$\hss}\sgchartC}
\def\sgchartC#1{\ifx,#1,\else
   \strut\vrule\kern-.4pt\hbox to\tcolw{\hss$#1$\hss}\expandafter\sgchartC\fi}
\def\sgchartD#1#2,{\ifx\relax#1\else\hbox to\tcolw{\hss$#1#2$\hss}\expandafter\sgchartD\fi}
\def\sgchartE#1#2,{\ifx\relax#1\else
    \ifx~#1\sgchartS\tcolw\circ \else\sgchartS\tcolw\bullet\fi \expandafter\sgchartE\fi}
\def\sgchartR{\leaders\vrule height2.8pt depth-2.4pt\hfil}
\def\sgchartS#1#2{\hbox to#1{\kern-.2pt\sgchartR \ifx\relax#2\else
   \kern-.7pt$#2$\kern-.7pt\sgchartR\fi\kern-.2pt}}
\ecatcode
%++++++++++++++++++++++++++++++++++++++++
\title{
    \vspace{2in}
    \textmd{\textbf{\labTitle}}
    \normalsize\vspace{0.1in}\\
    \vspace{0.1in}\large{\text{\class: \professor}}
    \vspace{3in}
}

\author{\name}
\date{Due: \dueDate}

\begin{document}
    \maketitle
    \thispagestyle{empty}
    \pagebreak
    \section{Introduction}
    \section{Data}
        \begin{table}[H]
            \centering
            \begin{tabular}{|c|c|c|}
                \hline
                \textbf{Measurement} & \textbf{Variable} & \textbf{Value} \\
                \hline
                Moment of Inertia of Windmill & $I$ & \SI{0.10066}{\kilogram\meter^2} \\
                Mass of Car 1 & $m_1$ & \SI{0.284}{\kilogram} \\
                Mass of Car 2 & $m_2$ & \SI{0.534}{\kilogram} \\
                Distance Traveled & $d$ & \SI{0.5}{\meter} \\
                Radius of Windmill & $r$ & \SI{0.325}{\meter} \\
                \hline
            \end{tabular}
            \caption{Measured Constants}
            \label{tab:constants}
        \end{table}
        \begin{table}[H]
            \centering
            \begin{tabular}{|c|c|c|}
                \hline
                \multicolumn{3}{|c|}{\textbf{Mass of Car (kg): 0.284}} \\
                \hline
                \textbf{Trial} & \textbf{Time from Release to Impact (s)} & \textbf{Period (s/rev)} \\
                \hline
                1 & 0.68 & 12.99 \\
                2 & 0.81 & 11.74 \\
                3 & 0.81 & 12.16 \\
                \hline
                \textbf{Average} & 0.767 & 12.297 \\
                \hline
            \end{tabular}
            \caption{Car 1 Data}
            \label{tab:car1} 
        \end{table}
        \begin{table}[H]
            \centering
            \begin{tabular}{|c|c|c|}
                \hline
                \multicolumn{3}{|c|}{\textbf{Mass of Car (kg): 0.534}} \\
                \hline
                \textbf{Trial} & \textbf{Time from Release to Impact (s)} & \textbf{Period (s/rev)} \\
                \hline
                1 & 1.01 & 9.51 \\
                2 & 1.01 & 9.43 \\
                3 & 1.03 & 8.26 \\
                \hline
                \textbf{Average} & 1.017 & 9.067 \\
                \hline
            \end{tabular}
            \caption{Car 2 Data}\label{tab:car2} 
        \end{table}
    \section{Analysis}
    Let variables with subscript 1 denote the first car and variables with subscript 2 denote the second car.
    \subsection{Observational Windmill Speed}
    The rotational speed of the windmill, $\omega$, is given by:
    \begin{align}
        \omega_1 = \frac{2\pi}{T} = \frac{2\pi}{12.297} = \text{\SI{0.511}{\frac{\radian}{\second}}} && \omega_2 = \frac{2\pi}{9.067} = \text{\SI{0.693}{\frac{\radian}{\second}}}\label{eq:windmillSpeedObservational}
    \end{align}
    \subsection{Theoretical Windmill Speed}
    We assume the car is moving on a frictionless surface. Therefore, the acceleration is constant meaning the velocity is given by:
    \begin{align}
        v_1 = \frac{\Delta d}{\Delta t} = \frac{0.5}{0.767} = \text{\SI{0.652}{\frac{\meter}{\second}}} && v_2 = \frac{\Delta d}{\Delta t} = \frac{0.5}{1.017} = \text{\SI{0.492}{\frac{\meter}{\second}}}\label{eq:carVelocity}
    \end{align}
    The linear momentum of the car is given by:
    \begin{align}
        p_1 = m_1 v_1 = 0.284(0.652) = \text{\SI{0.185}{\newton\second}} && p_2 = m_2 v_2 = 0.534(0.492) = \text{\SI{0.263}{\newton\second}}\label{eq:carMomentum}
    \end{align}
    The angle the car makes with the horizontal is $\theta = 90^\circ$. The angular momentum of the car is given by:
    \begin{align}
        L_1 &= p_1 r \sin \theta = 0.185(0.325) = \text{\SI{0.060}{\newton\meter\second}} \nonumber\\
        L_2 &= p_2 r \sin \theta = 0.263(0.325) = \text{\SI{0.085}{\newton\meter\second}}\label{eq:carAngularMomentum}
    \end{align}
    We assume that momentum is conserved in the system. Thus, after impact and coming to a complete stop, the momentum from the car is transferred to the windmill:
    \begin{equation}
        L_\text{car} = L_\text{windmill} = I\omega \label{eq:momentumConservation}
    \end{equation}
    We can solve for $\omega$:
    \begin{align}
        \omega_1 = \frac{L_1}{I} = \frac{0.060}{0.10066} = \text{\SI{0.59606}{\frac{\radian}{\second}}} && \omega_2 = \frac{L_2}{I} = \frac{0.085}{0.10066} = \text{\SI{0.84442}{\frac{\radian}{\second}}}\label{eq:windmillSpeedTheoretical}
    \end{align}
    \section{Conclusion}
    The percent loss of momentum from the car to the windmill is given by:
    \begin{equation}
        \delta = \frac{\abs*{L_\text{car} - L_\text{windmill}}}{L_\text{car}} \times 100
    \end{equation}
    where $L_\text{windmill}$ is the angular momentum using the observational windmill speed from \autoref{eq:windmillSpeedObservational}.
    \begin{align}
        \delta_1 &= \frac{\abs*{L_1 - I\omega_1}}{L_1} \times 100 = \frac{\abs*{0.060 - 0.10066(0.511)}}{0.060} \times 100 = 14.27\% \nonumber\\
        \delta_2 &= \frac{\abs*{L_2 - I\omega_2}}{L_2} \times 100 = \frac{\abs*{0.085 - 0.10066(0.693)}}{0.085} \times 100 = 17.93\%
    \end{align}
\end{document}