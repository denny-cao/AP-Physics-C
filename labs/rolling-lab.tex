%++++++++++++++++++++++++++++++++++++++++
\documentclass[article, 11pt]{article}
\usepackage{float}
\usepackage{setspace}
\usepackage{tabu} % extra features for tabular environment
\usepackage{amsmath}  % improve math presentation
\usepackage{graphicx} % takes care of graphic including machinery
\usepackage[margin=1in]{geometry} % decreases margins
\usepackage{cite} % takes care of citations
\usepackage[final]{hyperref} % adds hyper links inside the generated pdf file
\usepackage{tikz}
\usepackage{caption} 
\usepackage{fancyhdr}
\usepackage{amssymb} % symbols like /therefore
\usepackage{amsthm} % proofs
\usepackage{enumerate} % lettered lists
\usepackage{mathtools} % macros
\usetikzlibrary{scopes}
\usepackage{xcolor} \pagecolor[rgb]{0.12549019607,0.1294117647,0.13725490196} \color[rgb]{0.82352941176,0.76862745098,0.62745098039} % dark theme
\hypersetup{
	colorlinks=true,       % false: boxed links; true: colored links
	linkcolor=blue,        % color of internal links
	citecolor=blue,        % color of links to bibliography
	filecolor=magenta,     % color of file links
	urlcolor=blue         
}
\usepackage{titling}
\renewcommand\maketitlehooka{\null\mbox{}\vfill}
\renewcommand\maketitlehookd{\vfill\null}
\usepackage{siunitx} % units
\usepackage{verbatim} 
\newcommand{\labTitle}{Spinning Things Lab: Rollin' Down the Railway}
\newcommand{\class}{AP Physics C}
\newcommand{\professor}{Mr. Perkins}
\newcommand{\name}{Denny Cao}
\pagestyle{fancy}
\fancyhf{}% clears all header and footer fields
\fancyfoot[C]{--~\thepage~--}
\renewcommand*{\headrulewidth}{0.4pt}
\renewcommand*{\footrulewidth}{0pt}
\lhead{\name}
\chead{\class: \labTitle}
\rhead{\professor}


\fancypagestyle{plain}{%
  \fancyhf{}% clears all header and footer fields
  \fancyfoot[C]{--~\thepage~--}%
  \renewcommand*{\headrulewidth}{0pt}%
  \renewcommand*{\footrulewidth}{0pt}%
}

% Shortcuts
\DeclarePairedDelimiter\ceil{\lceil}{\rceil} % ceil function
\DeclarePairedDelimiter\floor{\lfloor}{\rfloor} % floor function

\DeclarePairedDelimiter\paren{(}{)} % parenthesis

\newcommand{\df}{\displaystyle\frac} % displaystyle fraction
\newcommand{\qeq}{\overset{?}{=}} % questionable equality

\newcommand{\Mod}[1]{\;\mathrm{mod}\; #1} % modulo operator

% Sets
\DeclarePairedDelimiter\set{\{}{\}}
\newcommand{\unite}{\cup}
\newcommand{\inter}{\cap}

\newcommand{\reals}{\mathbb{R}} % real numbers: textbook is Z^+ and 0
\newcommand{\ints}{\mathbb{Z}}
\newcommand{\nats}{\mathbb{N}}
\newcommand{\rats}{\mathbb{Q}}


% Counting
\newcommand\perm[2][^n]{\prescript{#1\mkern-2.5mu}{}P_{#2}}
\newcommand\comb[2][^n]{\prescript{#1\mkern-0.5mu}{}C_{#2}}

\setlength\parindent{0pt}

% Sign Charts
\newdimen\tcolw \tcolw=2.5em % the column width
\edef\ecatcode{\catcode`&=\the\catcode`&\relax}\catcode`&=4
\def\sgchart#1#2{\vbox{\offinterlineskip\halign{\hfil##\quad&##\hfil\crcr\sgchartA#2,:,%
   \omit\sgchartR&\kern.2pt\sgchartS{.5\tcolw}\relax\sgchartE#1,\relax,%
   \sgchartS{.5\tcolw}\relax\cr
   \noalign{\kern2pt}&\def~{}\kern.5\tcolw\sgchartD#1,\relax,\cr}}}
\def\sgchartA#1:#2,{\cr\ifx,#1,\else $#1$&\sgchartB#2{}\expandafter\sgchartA\fi}
\def\sgchartB#1{\hbox to\tcolw{\hss$#1$\hss}\sgchartC}
\def\sgchartC#1{\ifx,#1,\else
   \strut\vrule\kern-.4pt\hbox to\tcolw{\hss$#1$\hss}\expandafter\sgchartC\fi}
\def\sgchartD#1#2,{\ifx\relax#1\else\hbox to\tcolw{\hss$#1#2$\hss}\expandafter\sgchartD\fi}
\def\sgchartE#1#2,{\ifx\relax#1\else
    \ifx~#1\sgchartS\tcolw\circ \else\sgchartS\tcolw\bullet\fi \expandafter\sgchartE\fi}
\def\sgchartR{\leaders\vrule height2.8pt depth-2.4pt\hfil}
\def\sgchartS#1#2{\hbox to#1{\kern-.2pt\sgchartR \ifx\relax#2\else
   \kern-.7pt$#2$\kern-.7pt\sgchartR\fi\kern-.2pt}}
\ecatcode
%++++++++++++++++++++++++++++++++++++++++
\title{
    \vspace{2in}
    \textmd{\textbf{\labTitle}}
    \normalsize\vspace{0.1in}\\
    \vspace{0.1in}\large{\text{\class: \professor}}
    \vspace{3in}
}

\author{\name}
\date{November 30, 2022}

\begin{document}
    \maketitle
    \thispagestyle{empty}
    \pagebreak
    \section{Procedure}
    We placed a ramp of length \SI{1.2}{\meter} at a \SI{5}{\degree} incline, we rolled down three objects: a cart, disk, and ring. We tracked the time it took for each object to reach the bottom of the ramp.
    \section{Data and Analysis}
    To calculate the acceleraton of each object, we used $a = \df{2d}{t^2}$. \\
    \\
    To calculate the moment of inertia of each object, we used the following formulas:
    \begin{itemize}
        \item $I = mr^2$ for a ring
        \item $I = \df{1}{2}mr^2$ for a disk
    \end{itemize}
    From our set up, $d$ = \SI{1.2}{\meter} and $\theta$ = \SI{5}{\degree}.
    \begin{figure}[H]
        \centering
        \begin{tabular}{c|c|c|c|c|c}
            Object & Mass (\SI{}{\kilogram}) & Radius (\SI{}{\meter}) & Time (\SI{}{\second}) & Acceleration (\SI{}{\meter/\second^2}) & Moment of Inertia (\SI{}{\kilogram\meter^2})\\
            \hline
            Cart & 0.26 & & 1.6 & 0.938 & \\
            Disk & 0.63 & 0.075 & 1.9 & 0.665 & 0.0018 \\
            Ring & 0.63 & 0.075 & 2.3 & 0.454 & 0.0035 \\
        \end{tabular}    
    \end{figure}
    \subsection{Theoretical Acceleration}
    We calculated the theoretical acceleration of each object using the following equation:
    \begin{align*}
        \begin{aligned}
            a_\text{cart} &= g\sin{\theta} \\ &= (9.81)\sin(5^{\circ}) \\
            &\approx \SI{0.855}{\meter/\second^2} \\
            \\
            \\
        \end{aligned} \quad
        \begin{aligned}
            a_\text{disk} &= \frac{mgr^2\sin\theta}{mr^2 + \frac{1}{2}mr^2} \\
            &= \frac{2g\sin\theta}{3} \\
            &= \frac{2(9.81)\sin(5^{\circ})}{3} \\
            &\approx \SI{0.570}{\meter/\second^2}    
        \end{aligned} \quad
        \begin{aligned}
            a_\text{ring} &= \frac{mgr^2\sin\theta}{mr^2 + mr^2} \\
            &= \frac{g\sin\theta}{2} \\
            &= \frac{(9.81)\sin(5^{\circ})}{2} \\
            &\approx \SI{0.427}{\meter/\second^2}    
        \end{aligned}
    \end{align*}
    \subsection{Percent Error}
    We can calculate the percent error of each object by using the following equation:
    \begin{equation*}
        \delta = \frac{a_A - a_T}{a_T} \times 100\%
    \end{equation*}
    where $a_A$ is the recorded acceleration and $a_T$ is the theoretical acceleration. \\
    \\
    The percent error for each object is:
    \begin{itemize}
        \item Cart: 9.7\%
        \item Disk: 16.6\%
        \item Ring: 6.3\%
    \end{itemize}
\end{document}
