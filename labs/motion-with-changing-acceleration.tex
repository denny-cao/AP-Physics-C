\documentclass{article}
\usepackage{amsmath}

\begin{document}
\title{Lab: Motion with Changing Acceleration}
\author{Denny Cao}

\maketitle

\section{Procedure}
The objective of the lab was to investigate the impact air resistance has on a free-falling object, 
we dropped two objects. One was a ball which acted as a control, as the ball drops with little to no air resistance. 
We also dropped a paper cone, in which air resistance greatly impacts the time it takes to reach its terminal velocity—free-fall.
We made sure to measure the mass of the paper cone, $m = 0.005 \text{ kg}$, to use in calculations later on \\

\noindent
To collect data, we utilized the app Tracker to measure both the $y$-position in increments of $\Delta t$, 8.08 seconds.
The app allowed us to track the velocity and the change in velocity—acceleration—for each $t$-step. \\

\noindent
We then graphed the data, graphing both position v. time and velocity v. time to 
determine value of the acceleration due to gravity, $g$.

\section{Data and Calculations}
Upon finding the best regression model to use, 
we used the velocity model of the cone to determine the terminal velocity, $v_\text{term}$. \\

\noindent
We can model the situation with:
$$
F_\text{net} = ma = mg - F_\text{air} \\
$$

\noindent
where $F_\text{air}$ is the air resistance. We have two possible models of $F_\text{air}$:
\begin{align*}
    F_\text{air} = kv && F_\text{air} = kv^2
\end{align*}

\noindent
where $k$ is some constant. \\

\noindent
We can determine $k$ for both models by using $v_\text{term}$, the value of $v$ as $t$ approaches $\infty$:
\begin{align*}
    ma &= mg - k_1v & ma &= mg - k_2v^2 \\
    \lim_{t \rightarrow \infty}{ma} &= \lim_{t \rightarrow \infty}{mg - k_1v} & \lim_{t \rightarrow \infty}{ma} &= \lim_{t \rightarrow \infty}{mg - k_2v^2} \\
    0 &= \lim_{t \rightarrow \infty}{mg - k_1v} & 0 &= \lim_{t \rightarrow \infty}{mg - k_2v^2} \\
    -mg &= k_1v_\text{term} & -mg &= k_2v_{\text{term}}^2 \\
    \frac{-0.005 \times 9.8}{v_\text{term}} &= k_1 & \frac{-0.005 \times 9.8}{v_{\text{term}}^2} &= k \\
\end{align*}


\section{Conclusions}

\end{document}